\section{Comportamento}
Per offrire un esperienza gratificante all'utente si è preso in considerazione l'utilizzo della tecnologia JavaScript. In particolare sono stati definiti i seguenti script:\\
\begin{itemize}

	\item{avventure.js}: ha il compito di far comparire o nascondere la descrizione di una delle avventure una volta premuto sul div corrispondente;
	\item{checkInputLogin.js}: controlla che i campi del form di login dell'utente siano stati compilati, e nel caso di input vuoti colora di rosso il loro \textit{border};
	\item{checkInputRegistrazione.js}: controlla che i campi del form di registrazione dell'utente siano stati compilati, e nel caso di input vuoti colora di rosso il loro \textit{border};
	\item{showImage.js}: permette di mostrare l'immagine relativa ad ogni carta.  A causa del limitato spazio nel server universitario, non è stato possibile caricare le foto di ogni singola carta. Per ovviare al problema sono stati inseriti dei link a fonti esterne che mostrano la carta in lingua originale (inglese);
	\item{unselectradio.js}: rende possibile deselezionare un \textit{radiobutton} precedentemente selezionato.
\end{itemize}


\newpage