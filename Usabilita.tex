\section{Usabilità}
Particolare attenzione è stata posta all'usabilità del sito per garantire una miglior esperienza agli utenti. A tale scopo sono stati utilizzati:
\begin{itemize}
	\item \textbf{Navbar}: viene sempre evidenziata la voce della pagina in cui si trova l'utente all'interno della barra di navigazione. Essa non è cliccabile, evitando cosi un aggiornamento inutile;
	\item \textbf{Breadcrumbs}: per evitare che un utente con disabilità posso perdersi all'interno del sito, è stato riportato sotto la barra di navigazione il percorso effettuato dalla homepage. Per la normale utenza esso è stato sostituito parzialmente dalla barra di navigazione (vedi punto precedente);
	\item \textbf{Link}: ogni link presente nelle pagine è stato colorato in modo diverso dal normale testo, per renderlo immediatamente distinguibile dall'utente;
	\item \textbf{Foto}: ogni immagine è nitida e ben visibile. A causa del limitato spazio nel server universitario, non è stato possibile caricare le foto di ogni singola carta. Per ovviare al problema sono stati inseriti dei link a fonti esterne che mostrano la carta in lingua originale (inglese).
\end{itemize}


\newpage