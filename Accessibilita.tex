\section{Accessibilità}
\subsection{Separazione tra struttura, presentazione e comportamento}
Per migliorare l'accesso al sito a qualsiasi categoria di utenti è stata mantenuta la separazione tra struttura, presentazione e comportamento. \\
La prima è stata sviluppata tramite documenti HTML5 che richiamano i fogli di stile esterni CSS, che implementano la presentazione, e script esterni di JavaScript che ne determinano il comportamento. Nel caso in cui un utente non abiliti JavaScript il sito resta comunque accessibile.
\subsection{Tag meta}
In ogni pagina sono stati inseriti i seguenti tag meta:
\begin{itemize}
	\item \textbf{title}: titolo della pagina web;
	\item \textbf{description}: descrizione sintetica della pagina web;
	\item \textbf{keywords}: lista di parole chiave che permette di specificare gli argomenti trattati nella pagina web. 
\end{itemize}
\subsection{Screen reader}
Ogni elemento <img> presenta l'attributo \textit{alt} che descrive in maniera sintetica ciò che l'immagine ritrae o il suo scopo. \\ Mai sono state utilizzate immagini per riportare il testo, rendendo il contenuto informativo accessibile anche agli utenti non vedenti che quindi utilizzano gli screen reader. In alcuni casi, come ad esempio il logo del sito, il testo è stato nascosto e sostituito da un'immagine attraverso i fogli di stile CSS.

\subsection{Facilitazione alla navigazione}
Al fine di agevolare la visita al sito da parte degli utenti con disabilità si sono predisposte le seguenti facilitazioni:
\begin{itemize}
	\item \textbf{Link per spostarsi al contenuto}: prima della barra di navigazione è stato inserito un link nascosto, che permette agli utenti che utilizzano uno screen reader di passare direttamente al contenuto saltando la navbar;
	\item \textbf{Link per tornare alla barra di navigazione}: ...........
\end{itemize}
\newpage