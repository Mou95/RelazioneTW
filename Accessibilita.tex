\section{Accessibilità}
\subsection{Separazione tra struttura, presentazione e comportamento}
Per migliorare l'accesso al sito a qualsiasi categoria di utenti è stata mantenuta la separazione tra struttura, presentazione e comportamento. \\
La prima è stata sviluppata tramite documenti HTML5. Questi richiamano i fogli di stile esterni CSS, che implementano la presentazione, e gli script esterni di JavaScript che ne determinano il comportamento.
\subsection{Tag meta}
Per ogni pagina web sono stati inseriti i seguenti tag meta:
\begin{itemize}
	\item \textbf{title}: titolo;
	\item \textbf{description}: descrizione sintetica;
	\item \textbf{keywords}: lista di parole chiave che permette di specificare gli argomenti trattati. 
\end{itemize}
\subsection{Screen reader}
Ogni elemento img presenta l'attributo \textit{alt} che descrive in maniera sintetica ciò che l'immagine ritrae o quello che rappresenta. \\ 
Non sono state utilizzate immagini per riportare il testo, rendendo il contenuto informativo accessibile anche agli utenti che utilizzano gli screen reader. In alcuni casi, come ad esempio il logo del sito, il testo è stato nascosto e sostituito da un'immagine attraverso i fogli di stile CSS.\\
Per facilitare la lettura delle tabelle, sono stati inseriti i seguenti attributi:
\begin{itemize}
	\item{\textbf{id}}: inserito nell'elemento \textit{th} che lo identifica e lo rende riconoscibile;
	\item{\textbf{header}}: inserito negli elementi \textit{td} i quali fanno riferimento ai valori degli attributi \textit{id} dei \textit{th}.
\end{itemize}
Non è stato inserito l'attributo summary perchè obsoleto in HTML5.

\subsection{Facilitazione alla navigazione}
Al fine di migliorare l'esperienza d'uso degli utenti sono state inserite le seguenti facilitazioni:
\begin{itemize}
	\item \textbf{Link per spostarsi al contenuto}: prima della barra di navigazione è stato inserito un link nascosto per saltarla, permettendo agli utenti che utilizzano uno screen reader di passare direttamente al contenuto;
	\item \textbf{Link per tornare all'inizio del contenuto}: nelle pagine dove il contenuto visualizzato risulta essere troppo lungo è stato predisposto un link che permette di ritornare all'inizio del contenuto stesso. Il link viene rappresentato con una freccia rivolta verso l'alto attraverso l'utilizzo di CSS.
\end{itemize}
\newpage