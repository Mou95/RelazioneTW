\section{Struttura}
Nella cartella \textit{html} sono presenti i file delle parti statiche delle pagine con estensione \textit{.html}.\\
L'intero progetto è stato realizzato seguendo lo standard HTML5 utilizzando la sintassi di tipo XML. Alcune funzionalità introdotte con questo standard, tra cui \textit{required} ed \textit{autocomplete} sono state gestite anche a lato server nel caso il browser non le supporti.\\
<<<<<<< HEAD
Il gruppo è in accordo a considerare i tag \textit{header, nav} e \textit{footer} ormai supportati da tutti i browser più recenti (IE 9+).
Per ogni pagina 'x' sono stati definiti una pagina 'x.html', che rappresenta la parte statica del contenuto, ed un file 'x\textunderscore head.html' che contiene il tag head ed i rispettivi metatag. Inoltre, visto che header e footer sono condivisi da tutte le pagine, sono stati scorporati in due file a sè stanti ed inclusi nelle pagine sopracitate. \\
=======
Il gruppo è in accordo a considerare i tag \textit{header, nav} e \textit{footer} ormai supportati da tutti i browser più recenti (IE 9+).\\
Per ogni pagina 'x' sono stati definiti una pagina 'x.html', che rappresenta la parte statica del contenuto, ed un file 'x\textunderscore head.html' che contiene il tag head ed i rispettivi metatag. Inoltre, visto che header e footer sono condivisi da tutte le pagine, sono stati scorporati in due file a sé stanti ed inclusi nelle pagine sopracitate. \\
>>>>>>> 0f0dc9c106c7b38f1aa2eadd318290a91d5f35ee
A questa regola fa eccezione solo la pagina di errore 404, la quale è interamente statica, e quindi è contenuta in un unico file statico 'errore.html'.


\newpage