\section{PHP}
PHP viene usato per gestire la visualizzazione e la gestione delle informazioni dinamiche.

\subsection{Utente non autenticato}
L'utente non autenticato ha accesso alle seguenti pagine ed ai rispettivi contenuti:
\begin{itemize}
	\item \textbf{avventure.php}: vengono elencate le varie avventure del gioco;
	\item \textbf{carte.php}: permette di visualizzare ed effettuare ricerche relative alle carte;
	\item \textbf{discussione.php}: permette di visualizzare titolo, testo e commenti di un topic inserito da un utente;
	\item \textbf{forum.php}: permette di visualizzare le varie sezioni del forum;
	\item \textbf{guide.php}: visualizza le categorie delle guide disponibili;
	\item \textbf{guideEroe.php}: visualizza le guide relative ad uno specifico eroe;
	\item \textbf{index.php}: visualizza i migliori mazzi, le ultime guide inserite e un consiglio scelto in maniera casuale;
	\item \textbf{login.php}: permette all'utente di effettuare l'accesso;
	\item \textbf{mazzi.php}: visualizza i mazzi inseriti dagli utenti e permette di effettuare ricerche su di essi;
	\item \textbf{mostraGuida.php}: visualizza titolo e testo di una guida inserita dall'utente;
	\item \textbf{mostraMazzo.php}: visualizza le informazioni relative ad un mazzo (nome, voti, carte, utente);
	\item \textbf{registrazione.php}: permette all'utente di diventare un membro del sito;
	\item \textbf{ricercaForum.php}: visualizza i risultati di una ricerca effettuata in forum;
	\item \textbf{ricercaGuida.php}: visualizza i risultati di una ricerca effettuata in guide;
	\item \textbf{sezioneForum.php}: visualizza i topic di una sezione specifica;
\end{itemize}
\subsection{Utente autenticato}
Dopo aver effettuato l'accesso, l'utente può accedere inoltre alle seguenti funzionalità:
\begin{itemize}
	\item  \textbf{aggGuida.php}: offre la possibilità di inserire una guida;
	\item \textbf{creaMazzo.php}: offre la possibilità di inserire un mazzo;
	\item \textbf{scegliClasse.php}: permette di selezionare la classe del mazzo che si vuole creare;
	\item \textbf{user.php}: visualizza il profilo dell'utente permettendogli di visualizzare ed eliminare i mazzi e le guide da lui create
\end{itemize}
Inoltre viene offerta la possibilità di creare e commentare mazzi e topic.
\subsection{Amministratore}
L'utente amministratore (admin) gode delle stesse funzionalità dell'utente autenticato. Inoltre, ha la possibilità di eliminare qualsiasi mazzo, topic, commento e/o guida.
\subsection{Gestione della sessione}
Se il login avviene con successo viene creata una sessione sfruttando \textdollar \textunderscore SESSION. Utilizzando suddetta variabile vengono effettuati controlli per determinare il tipo di utente e l'insieme delle funzionalità a cui può accedere.

\newpage