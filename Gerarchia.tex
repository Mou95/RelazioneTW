\section{Gerarchia dei file}
La gerarchia dei file è così composta:
\begin{itemize}
	\item \textbf{Radice}: nella root directory sono presenti tutti gli script PHP delle pagine a cui l'utente può accedere e le seguenti cartelle:
	\begin{itemize}
		\item \textbf{css}: contiene tutti i fogli di stile;
		\item \textbf{html}: contiene la parte statica di struttura delle pagine con eventuali etichette che verranno rimpiazzate dinamicamente dai vari script PHP;
		\item \textbf{images}: contiene le immagini utilizzate nel sito e alcune sottocartelle:
		\begin{itemize}
			\item{icon}: immagini che sono icone nel nostro sito;
			\item{heroes}: immagini relative agli eroi di HearthStone;
			\item{user}: immagini del profilo degli utenti.
		\end{itemize}
		\item \textbf{php}: che si suddivide in:
		\begin{itemize}
			\item {Database}: contiene gli script che effettuano le operazioni sul database. Attravverso PDO si è cercato di costruire un'astrazione tra le classi contenute in Page e il database;
			\item {Page}: contiene l'interfaccia Page(Page.php) che viene implementata dalle classi all'interno degli altri script. Le classi forniscono dei metodi per la creazione dinamica delle pagine e la gestione dei dati.
		\end{itemize}
		\item \textbf{script}: contiene i vari script realizzati in JavaScript;
	\end{itemize}
\end{itemize}


\newpage